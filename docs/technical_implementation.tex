\documentclass[]{article}

\usepackage{fullpage}
\usepackage{physics}
\usepackage{amsmath}
\usepackage{amssymb}
\usepackage{url}
\usepackage{comment}

%opening
\title{Implementation Notes for Gyrokinetic Particle Pusher}
\author{Norman M. Cao}

\begin{document}

\maketitle

\section{Equations of motion and geometry}

\subsection{Gyrokinetic characteristic equations}
We aim to push particles in electrostatic gyrokinetics.
The equations of motion are given by
\begin{subequations}
	\begin{gather}
		B_\parallel^* \dot{\vb{R}} = \frac{1}{q} \vu{b} \times \nabla H + v_\parallel \vb{B}^* \\
		B_\parallel^* \dot{p}_\parallel = -\vb{B}^* \cdot \nabla H \\
		\dot{\mu} = 0
	\end{gather}
\end{subequations}
With some definitions
\begin{subequations}
	\begin{gather}
		\vu{b} := \vb{B} / B \\
		\vb{B}^* := \vb{B} + \nabla \times (p_\parallel \vu{b} / q) \\
		B_\parallel^* := \vu{b} \cdot \vb{B}
	\end{gather}
\end{subequations}
\begin{subequations}
	\begin{gather}
		H = p_\parallel^2 / 2 m + \mu B + q \mathcal{J}[\Phi] \\
		v_\parallel := \partial_{p_\parallel} H = p_\parallel / m
	\end{gather}
\end{subequations}
and \(m,q\) are the species mass and charge respectively.

\subsection{Cylindrical coordinates}
We primarily work in a right-handed \((R,\varphi,Z)\) cylindrical coordinate system.
Thus, \(\varphi\) points into the \((R,Z)\) plane.
Recall that for a path \(\vb{R} = (R(t), \varphi(t), Z(t))\), we have that
\begin{gather*}
	\dot{\vb{R}} = \dot{R} \vu{R} + \dot{Z} \vu{Z} + R \dot{\varphi} \vu*{\varphi} \\
	\ddot{\vb{R}} = (\ddot{R} - R \dot{\varphi}^2) \vu{R} + \ddot{Z} \vu{Z} + (R \ddot{\varphi} + 2 \dot{R} \dot{\varphi}) \vu*{\varphi}
\end{gather*}
Typically we will work with the orthonormal basis of unit vectors \((\vu{R}, \vu*{\varphi}, \vu{Z})\).
From the above expressions, the velocity acts on coordinates as
\begin{equation*}
	\dot{R} = \dot{\vb{R}} \cdot \vu{R} \qquad
	\dot{Z} = \dot{\vb{R}} \cdot \vu{Z} \qquad
	\dot{\varphi} = (\dot{\vb{R}} \cdot \vu*{\varphi}) / R
\end{equation*}
then, the acceleration acts on components of the velocity vector (in the orthonormal basis) as
\begin{gather*}
	 \\
	(\dot{\vb{R}} \cdot \vu{R})' = \ddot{\vb{R}}\cdot \vu{R} + R \dot{\varphi}^2 \\
	(\dot{\vb{R}} \cdot \vu{Z})' = \ddot{\vb{R}} \cdot \vu{Z} \\
	(\dot{\vb{R}} \cdot \vu*{\varphi})' = \dot{R} \dot{\varphi} + R \ddot{\varphi} = \ddot{\vb{R}} \cdot \vu*{\varphi} - \dot{R}\dot{\varphi}
\end{gather*}

\subsection{Magnetic field}
We use the following representation of the magnetic field and current
\begin{gather*}
	\vb{B} = F(\psi) \nabla \varphi + \nabla \varphi \times \nabla \psi = \frac{F(\psi) \vu*{\varphi} + \vu*{\varphi} \times \nabla \psi}{R} \\
	\nabla \times \vb{B} = F' \nabla \psi \times \nabla \varphi + \nabla \times (\nabla \varphi \times \nabla \psi)
\end{gather*}
where \(\psi\) is the poloidal flux.
Note the second term in the current can be evaluated using an in-plane curl

It's useful to have the following representations for certain terms in the gyrokinetic equation:
\begin{gather*}
	\nabla B = \frac{\nabla(RB) - B \nabla R}{R} \\
	2 RB \nabla(RB) = \nabla(R^2B^2) = \nabla(F^2 + |\nabla\psi|^2) = 2 F' \nabla \psi + 2 \operatorname{Hess}[\psi] \nabla \psi \\
	\nabla \times \vu{b} = \frac{B (\nabla \times \vb{B}) - (\nabla B) \times \vb{B}}{B^2}
\end{gather*}
note that these can be written purely in terms of analytic derivatives of \(\psi(R,Z)\) and \(F(\psi)\).

\section{\(C^1\) interpolation on unstructured meshes}

\subsection{Field line tracing}
Let \(\vec{R}_B(R,Z;\varphi) = (R_B(...), Z_B(...))\) be the motion of the field-line trace starting at \((R,Z)\) on a poloidal plane, parameterized by the toroidal angle \(\varphi\) moved along the field line.

\(\vec{R}_B\) satisfies the ODE
\begin{equation*}
	\dv{\vec{R}_B}{\varphi} = \vec{b}_p \circ \vec{R}_B := \eval{\frac{R\vb{B}_p}{B_t}}_{\vec{R}_B} = \eval{\frac{R \vu*{\varphi} \times \nabla \psi}{F(\psi)}}_{\vec{R}_B}; \qquad
	\vec{R}_B(R,Z;0) = (R,Z)
\end{equation*}
This ODE is essentially a reparameterization of the magnetic field line ODEs with \(\varphi\) as time.
For \(\dv{\varphi}\) we think of \((R,Z)\) as being parameters.
These ODEs have an associated variational equation
\begin{equation*}
	\dv{[D\vec{R}_B]}{\varphi} = ([D\vec{b}_p] \circ \vec{R}_B) [D\vec{R}_B]; \qquad D\vec{R}_B(R,Z;0) = I_{2\times2}
\end{equation*}
here we think of \(D\) as the differential in \((R,Z)\) with \(\varphi\) as a parameter, that is:
\begin{equation*}
	D\vec{b}_p = \begin{bmatrix}
		\partial_R(\vec{b}_p \cdot \vu{R}) & \partial_Z(\vec{b}_p \cdot \vu{R}) \\
		\partial_R(\vec{b}_p \cdot \vu{Z}) & \partial_Z(\vec{b}_p \cdot \vu{Z})
	\end{bmatrix}
\end{equation*}
\begin{equation*}
	D\vec{R}_B = \begin{bmatrix}
		\partial_R R_B & \partial_Z R_B \\
		\partial_R Z_B & \partial_Z Z_B
	\end{bmatrix}
\end{equation*}

\subsection{Field-aligned interpolation}
Suppose we are trying to interpolate \(\phi(R,\varphi,Z)\) knowing its values on some equally spaced poloidal planes \(\phi_i(R,Z)\).
This can be accomplished by
\begin{equation*}
	\phi(R,\varphi,Z) = \sum_{i} p_i(\varphi) \phi_i(\vec{R}_B(R,Z;\varphi_i-\varphi))
\end{equation*}
here \(p_i\) are some piecewise polynomial basis functions.
We can compute its gradient by
\begin{equation*}
	\nabla \phi = \sum_{i} \left[p' (\phi_i \circ \vec{R}_B) \nabla \varphi + p_i \nabla (\phi_i \circ \vec{R}_B)\right]
\end{equation*}
Using the chain rule,
\begin{align*}
	\nabla (\phi_i \circ \vec{R}_B) &=
	\begin{bmatrix}
		\nabla R & \nabla Z
	\end{bmatrix}
	[D(\phi_i \circ \vec{R}_B)]^T
	-
	\nabla \varphi \left(\dv{(\phi_i \circ \vec{R}_B)}{\varphi}\right) \\
	&=
	\begin{bmatrix}
		\vu{R} & \vu{Z}
	\end{bmatrix}
	[D \vec{R}_B]^T
	[[\nabla \phi_i] \circ \vec{R}_B]
	-
	\frac{\vu*{\varphi}}{R}\left([[\nabla \phi_i]\circ \vec{R}_B] \cdot \dv{\vec{R}_B}{\varphi}\right)
\end{align*}
Note that it's possible to show that \(\vb{B} \cdot \nabla (\phi_i \circ \vec{R}_B) = 0\).

\subsection{Choice of basis functions}

Traditional cubic spline interpolation, which minimizes `bending' and \(C^2\) continuity, has polynomial coefficients which are computed by solving a linear system involving all of the data points as well as boundary conditions.
Instead we rely on polynomial splines which involve only the 4 points in the neighborhood of any \(\varphi\), and generally enforce \(C^1\) continuity at the nodes.

Two options are considered.
The first is cubic Hermite interpolation, with an array of polynomial coefficients
\begin{equation*}
	p = \begin{bmatrix}
		0 & -1/2 & 1 & -1/2 \\
		1 & 0 & -5/2 & 3/2 \\
		0 & 1/2 & 2 & -3/2 \\
		0 & 0 & -1/2 & 1/2
	\end{bmatrix}
\end{equation*}
where \(p_i(t) = \sum_{j=0}^{3} p_{ij} t^j\) (here the array entries are being 0-indexed).
This scheme exactly interpolates the nodes and also enforces \(C^1\) continuity at the nodes with a value of the derivative given by the centered difference of the adjacent two nodes.

The second is a quadratic smoothing spline,
\begin{equation*}
	p = \begin{bmatrix}
		1/4 & -1/2 & 1/4 \\
		1/2 & 0 & -1/4 \\
		1/4 & 1/2 & -1/4 \\
		0 & 0 & 1/4
	\end{bmatrix}
\end{equation*}
This scheme can be thought of as the anti-derivative of linear interpolation on the derivative, computed via centered difference, while enforcing \(C^1\) continuity at the nodes.
The quadratic dependence sacrifices exact interpolation at the nodes in exchange for a derivative with less oscillations.

Finally we remark that parallel noise seems to be non-negligible; in theoretical cases, a Lanczos filter is applied along the field line to smooth out these high-frequency parallel fluctuations.


\subsection{Interpolation on poloidal planes}

Interpolation on poloidal planes uses rHCT elements, which are \(C^1\) elements that minimize a `bending energy'.
The code is essentially a fork of the matplotlib \texttt{CubicTriInterpolator}\footnote{\url{https://matplotlib.org/stable/api/tri_api.html#matplotlib.tri.CubicTriInterpolator}} with a few optimizations.

\section{Extraction of Ballooning Coefficients}

\subsection{Straight Field-line Coordinates}
On the closed flux surfaces, let \(\theta_g\) be the geometric poloidal angle relative to the magnetic axis, with \(\theta_g = 0\) representing the outboard midplane.
Using \(F(\psi) = B_t / R\), we can compute the relationship between the straight field-line angle \(\theta\) in terms of \(\theta_g\) by
\begin{gather*}
	\theta = \frac{1}{q(\psi)} \int_{\theta_0(\psi)}^{\theta_g} \frac{F(\psi)}{B_p(\psi,\theta_g')} \ell'(\theta_g') \dd{\theta_g'} \\
	q(\psi) = \frac{1}{2\pi} \int_{0}^{2\pi} \frac{F(\psi)}{B_p(\psi,\theta_g')} \ell'(\theta_g') \dd{\theta_g'}
\end{gather*}
where \(\ell'(\theta_g)\) is the derivative of the arclength of the field line along the flux surface with respect to \(\theta_g\), and \(\theta_0(\psi)\) is the arbitrary offset on each flux surface where \(\theta=0\) lies.
An easy choice is \(\theta_0(\psi) = 0\), which results in \(\theta=0\) being the outboard midplane.
This relationship is then numerically inverted to get \(\theta\).

\subsection{Ballooning Transform}

Moving to flux coordinates \((\psi, \zeta, \theta)\) with \(\zeta=\varphi\), we can always write
\begin{equation*}
	\phi(R,\varphi,Z) = \sum_{n=-\infty}^{\infty} e^{i n \zeta} \phi_n(\psi,\theta)
\end{equation*}
We can move to the covering space \(\theta \mapsto \eta\), let \(\rho = nq\), and introduce the eikonal factor
\begin{gather*}
	\phi_n(\psi, \theta) = \sum_{\ell=-\infty}^{\infty} \hat{\phi}_n(\psi, \eta + 2 \pi \ell) \\
	\hat{\phi}_n(\psi, \eta) = e^{-i\rho\eta} f_n(\rho,\eta) \\
	f_n(\rho, \eta) = \int_{-\infty}^{\infty} \dd{\theta_k} e^{i\rho\theta_k} \tilde{f}_n(\theta_k, \theta)
\end{gather*}
the complete representation is then
\begin{equation*}
	\phi(R,\varphi,Z) = \sum_{n=-\infty}^{\infty} \sum_{\ell = -\infty}^{\infty} \int_{-\infty}^{\infty} \dd{\theta}_k e^{in(\zeta - q(\eta - \theta_k + 2 \pi \ell))} \tilde{f}_n(\theta_k, \theta + 2 \pi \ell)
\end{equation*}

Now, if we had an exact ballooning symmetry, we would have for a fixed parameter \(\theta_k\) (related to the radial wavenumber),
\begin{equation*}
	f_n(\rho,\eta) = e^{i\rho\theta_k} \tilde{F}_n(\eta)
\end{equation*}
This is exactly analogous to \(f(x) = \tilde{F} e^{ikx}\) with \(k\) is a fixed parameter.
Our goal is to find an approximation to \(\tilde{f}_n\).

Taking \(S=n(\zeta - q \eta)\), we can compute
\begin{gather*}
	\nabla (e^{iS} f_n(\rho, \eta)) = e^{iS} \left[i\nabla S f_n + \partial_q f_n \nabla q + \partial_\eta f_n \nabla \eta\right] \\
	\nabla S = n(\nabla \zeta - q \nabla \eta - \eta \nabla q)
\end{gather*}

Note that in the ballooning representation, we also have
\begin{gather*}
	\tilde{S} = n(\zeta - q(\eta - \theta_k)) \\
	\nabla(e^{i\tilde{S}} \tilde{f}_n(\theta_k, \eta)) = e^{i\tilde{S}} (i\nabla \tilde{S} \tilde{f}_n + \partial_\eta \tilde{f}_n \nabla \eta) \\
	\nabla \tilde{S} = n(\nabla \zeta - q \nabla \eta - (\eta - \theta_k) \nabla q)
\end{gather*}

Note furthermore we have the gyroaverage operator taking the eikonal approximation
\begin{gather*}
	\ev{e^{i\tilde{S}}\tilde{f}_n(\theta_k, \eta)} \approx e^{i\tilde{S}} J_0(k_\perp \rho) \tilde{f}_n(\theta_k, \eta) \\
	\vb{k}_\perp = \nabla \tilde{S}
\end{gather*}

\subsection{Up-down Symmetry}
One final thing we do is to modify \(\theta_0\) to account for up-down symmetry.
We compute \(v_r = \vb{v}_D \cdot \vu{r}\) and \(v_y = \vb{v}_D \cdot \vu{y}\) where \(\vu{y} = \vu{r} \times \vu{b}\) is the binormal vector.
We take \(\theta_0 = \operatorname{argmax}_{\theta} v_y\).

\subsection{Approximation by Gauss-Hermite Functions}

Finally, we approximate \(f_n(\rho,\eta)\) (equivalently \(\tilde{f}_n(\theta_k, \eta)\)) with complex-amplitude Gauss-Hermite functions.

Note that naively, if we try to extract \(\phi_n\) from the data \(\phi\), aliasing issues cause pollution in the toroidal mode number spectrum.
Instead, we first upsample the toroidal resolution using field-aligned interpolation (rather than numerically following field lines, we use constant \(\alpha=\zeta - q \theta\)), perform the FFT, then truncate in frequency space.
Additionally, we apply a Lanczos filter in the parallel direction to reduce high-frequency parallel noise -- not sure what its source is.

To compute gradients of \(\phi\), note the coordinate transform

Define \(v_D^q,v_D^\alpha\) by
\begin{equation*}
	\vb{v}_D\cdot \nabla = v_D^q \partial_q + v_D^\alpha \partial_\alpha = v_D^R \partial_R + v_D^Z \partial_Z + v_D^\varphi \partial_\varphi
\end{equation*}
Note that \(v_D^\varphi = \vb{v}_D \cdot \vu*{\varphi} / R\).

We have the following coordinate transforms
\begin{equation*}
	q = q(\psi) \qquad \alpha = \zeta - q \theta \qquad \eta = \theta
\end{equation*}
\begin{equation*}
	\begin{bmatrix}
		\dd{q} \\ \dd{\alpha} \\ \dd{\eta}
	\end{bmatrix}
	=
	\begin{bmatrix}
		q' & 0 & 0 \\
		-q' \theta & 1 & -q \\
		0 & 0 & 1
	\end{bmatrix}
	\begin{bmatrix}
		\dd{\psi} \\ \dd{\zeta} \\ \dd{\theta}
	\end{bmatrix}
	=
	\begin{bmatrix}
		q' & 0 & 0 \\
		-q' \theta & 1 & -q \\
		0 & 0 & 1
	\end{bmatrix}
	\begin{bmatrix}
		\psi_R & 0 & \psi_Z \\
		0 & 1 & 0 \\
		\theta_R & 0 & \theta_Z
	\end{bmatrix}
	\begin{bmatrix}
		\dd{R} \\ \dd{\varphi} \\ \dd{Z}
	\end{bmatrix}
\end{equation*}

\end{document}
